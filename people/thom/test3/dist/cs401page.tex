\documentstyle[12pt,margins]{article}

\title{HPF Front-End and F90 Back-End to the Psi Compiler}
\author{Thom McMahon}
\date{December 15, 1994}

\begin{document}
\maketitle
\section{Introduction}
My project for this semester was to continue work on the interface between
the Psi Compiler and HPF, extending the subset of recognizable HPF 
grammar and providing for an interface between the output Fortran 90
code and MPI.  My goals were mostly fulfilled.  The part of HPF 
recognizable by the parser has grown, yet not as much as I had originally
intented.  However, although the MPI interface written as yet
doesn't work, it hopefully never will have to.  This interface is
an interface to the C version of MPI, since the Fortran version is
not currently available, but soon should be on at least the hypercube,
if not the Sun workstations.  Availiablity of Fortran MPI will
make any interface to C MPI unnecessary and probably wasted effort.
The interface to the C MPI was done so that programs could be run and
possibly benchmarked before the Fortran version of MPI was in place.

Since this is an ongoing project, I will briefly describe what work
was done over the summer, what work was done during the semester, and
what work needs to be done in the future.

\section{The Summer}
When work began on this project this summer, the front-end of the
Psi Compiler was an
interface to MOAL (Mathematics of Array Language) and the back-end was an
interface was to C.  The first goal of implementing an interface to
HPF was to separate these
interfaces from the rest of the code so that only a few modules would have
to be changed to create the HPF interface.  The code for the front-end
interface was mainly the parser and
lexical analyzer and therefore easy to isolate; however,
the back-end was more difficult, since much of the code to generate the
C output was spread throughout various parts of the compiler.  Two
modules (code.c and dist.c) housed the majority of the code generation,
while two other modules (ident.c and vect.c) housed a minor portion of
code generation.  These two other modules were stripped of their
code generation parts, which were placed in code.c.  

\samepage{
Once isolation was complete, the exisiting front-end interface was
replaced with a new parser and lexical analyzer for a subset of HPF, and
the back-end code generation modules were modified to output
Fortran 90.
}

\subsection{The Front-End}
The parser and lexical analyzer were created using tools
from Purdue, namely PCCTS (Purdue Compiler Construction Tool Set).  
PCCTS turned out to be fairly good as far as describing grammars, using
extended BNF for grammar descriptions; however,
it produces top-down LL grammars, as opposed to yacc or Bison which produce
more general bottom-up LALR grammars. In addition PCCTS has trouble 
accomplishing right recursion (something needed for right associativity
of expressions).

Once the parser was in place, two modules, hpfm\_func.c
and compat.c were created to act as an interface to the Psi Compiler
reduction engine.  Hpfm\_func.c contains the routines that convert
Fortran 90 intrinsic functions such as {\tt MATMUL} into their internal
Psi Calculus representations that are used by the Psi Compiler.
Compat.c contains 4 functions: make\_compatable,
reduce\_extended, code\_extended, and mark\_extended.  Make\_compatable
is called by the parser when it encounters an operator or intrinsic
function call, such as plus, times, MATMUL, etc....  Make\_compatable's
job is to make sure that the arguments (passed as root nodes of the
parse tree that represent parts of an expression)
are compatable with the operator
or function call.  For example, suppose we have a program with
the line:
\begin{verbatim}
  A = MATMUL (A**B, CSHIFT(TRANSPOSE (C), 3))
\end{verbatim}
{\tt MATMUL} AND {\tt CSHIFT} are supported by the Psi Compiler, whereas
at present {\tt TRANSPOSE} is not.  The parser, upon recognizing the
{\tt CSHIFT(TRANSPOSE (C), 3))} calls make\_compatable, which 
realizes that {\tt TRANSPOSE(C)} must be assigned to a temporary variable
that will be used as an argument to {\tt CSHIFT}.  
Make\_compatable then creates a node in the parse tree, marked to be 
of type {\tt EXTENDED}, and returns this node to the parser.  Events are
similar for the {\tt MATMUL}.  Make\_compatable realizes that the
{\tt CSHIFT} can be handled by the Psi Compiler, and thus is a valid
argument to {\tt MATMUL}, whereas {\tt A**B} is not.  Another {\tt EXTENDED}
node is created, and nothing is done to the {\tt CSHIFT} part.  

When the parse tree is being transgressed by the Psi Compiler during the
reduction phase, and a node
of type {\tt EXTENDED} is encountered, the compiler calls the 
reduce\_extended function.   This function takes care of adding nodes
to the list of reduced expressions that will handle the temporary
variable assignments discussed above.  When these nodes are encountered during
the code generation of the Psi Compiler, code\_extended is called to
write to the output Fortran 90 file. 

This whole process is actually much more complicated and confusing than
presented here, requiring a pretty thourough understanding of the inner
workings of the Psi Compiler in order to permit complete comprehension.  

\samepage{
\subsection{The Back-End}
The back-end was again more difficult to modify then the front-end,
since many things
that could be done in C were not allowed in Fortran 90.  Most problems were
related to the strong typing and limited pointer capability of the Fortran
language.  The worst problem was rearranging all of the code generation
to compensate for Fortran's column-major order for arrays, as opposed
to C's row-major order.  Put simply, everything had to be turned
upside-down and backwards relative to the way it was originally.  
}

\section{The semester}
For this semester, I concentrated on attempting to add to the 
subset of HPF recognized by the parser, as well as getting the compiler
to output Fortran 90 code for distributed programs that compiles and
runs correctly.

\subsection{Additional Grammar}
Adding things to the parser was fairly easy compared to the
rest.  There were however a few ambiguities of the input grammar, releated
to the syntax of the Fortran language itself, that caused problems.

For example, both
\begin{verbatim}
                DO I = 1, 10
                  A = B
                END DO
\end{verbatim}
and
\begin{verbatim}
                DO
                  I = 1
                  A = B
                END DO
\end{verbatim}
are perfectly fine in Fortran.  For an LL parser, they are a nightmare,
since it can not be determined whether the {\tt I = 1} is part of the
loop control or an assignment statement until the comma is parsed.  
This is one of the main reasons I will be experimenting with Bison,
which creates a LALR parser.

\samepage{
\subsection{The Rest}
Everything else depended upon proper working of the back-end
Psi Compiler.  Since the Psi Compiler was nowhere near perfect at the
beginning of the semester, barely approaching stability by the end of 
the semester, the isolation of errors was made extremely difficult; 
they could just as well be in new code as old.  However, 
progress did become more rapid as the semester progressed and more bugs
were fixed in the Psi Compiler.
}

\subsubsection{The Distributed Code}
The orignal distributed code was a little on the messy side, since it attempted
to be as general as possible.  This generality was not only confusing,
but caused many problems when converting the back-end to output Fortran
90.  Much of this output code was perfectly legal in C, but when
\samepage{
translated into Fortran 90, was perfectly {\it illegal}.  For example,
logical comparisons used in expressions such as
\begin{verbatim}
                A = C > B
\end{verbatim}
are valid in C, but invalid in Fortran.  Other invalid parts included
}
pointer arithmetic; trying to make pointers, declared as pointers to
real numbers, point to arrays of real numbers; and allocation of memory
for variables not declared in local scope (i.e. ones passed as function
parameters).  The solution to these problems was to eliminate the
generality from the
output code produced.  This, however, placed strict limitations on how
arrays could be distributed and what operations could be performed.
But until enough theory and experience exists to remove these restrictions,
they must remain.

The use of MPI was integrated
into the compiler, as an alternative to the socket code previously 
available. MPI is a standard message-passing interface with around 125
functions devoted to communication between processors for parallel
programs.  An attempt to create an interface to the only version of MPI
we currently have at UMR (the C version) was met with suprising 
difficulty.  Basically, there were around four layers of function
calls traversed in order to be able to use the MPI routines:
a C main program to call the functions
generated by the Psi Compiler that called routines in a Fortran module
that called routines in a C moudule that called the actual MPI functions.
The C main program was needed in order to access the command line
arguments.  The Fortran module provides a generic interface to whatever
sort of message passing routines we want to use (in this case MPI).  
Finally the C module was needed to interpret the way Fortran 90 represents
arrays (a structure that contains the array as well as shape and 
dimension information) into the way C represents an array (a pointer).

\section{Future Work}
This project is an ongoing one, and much work lies ahead.  Bison is
being investigated as an alternative parser generator for HPF.  I
believe that Bison will be able to create a much more general parser,
removing some of the ambiguities that occur in the current parser.  
The distributed code has been severely cut back, and must be built
back up in order to recover some of the lost generality without adding
and inordinate amount of excess code.  In addition, more of Fortran 90's
intrinsic functions such as the trigonometric functions, {\tt TRANSPOSE},
{\tt MOD}, and others need to be supported by the Psi Compiler.  This
would eliminate much of the temporary variable assignments currently
done.  Finally, and most importantly, {\bf all} of the code, for both the
HPF interface and the actual Psi Compiler needs to be {\bf documented}.
This has been neglected in the past, and can be ignored no further.
\end{document}
