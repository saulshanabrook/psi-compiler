\documentstyle[margins]{article}
\newcommand{\cat}{+\!\!\!\!+}
\newcommand{\take}{\,\bigtriangleup\,}
\newcommand{\drop}{\,\bigtriangledown\,}
\newcommand{\karate}{\!\widehat{\hbox to 10pt{}}\!}
\newcommand{\rshp}{\widehat{\rho}}
\newcommand{\one}{$<$\! 1\!$>$}
\newtheorem{definition}{Definition}
\newcommand{\Size}{\tau\,}
\newcommand{\Dim}{\delta\,}
\newcommand{\Ravel}{\,\mbox{\tt rav}\,}
\newcommand{\Reshape}{\,\widehat{\rho}\,}
\newcommand{\Reduce}{\,\mbox{\tt red}\,}

\title{Psi Compiler\\Programmer's Guide\\Distributed Augmentation}
\author{Scott Thibault\\Compiled by\\Thom McMahon}
\date{October 28, 1994}

\begin{document}
\maketitle
\section*{Intoduction}
A distributed array can be described by three vectors.  
Let e be the shape of local sub-array on each processor.
(ex $<$8 6$>$ array on 4 processors e = $<$2 6$>$)
Let g be the shape of the abstract processor array which is the
same dimension as the distributed array.
(ex $<$8 6$>$ array on 4 processors g = $<$4 1$>$) --
columns are not distributed row distributed over four processors  --
Let c be the shape of the cyclic structure of the array.
(ex $<$8 6$>$ array on 4 processors c = $<$1 1$>$, If you wanted to 
distribute the rows of a $<$32 6$>$ on 2 processors cyclicly, which the 
compiler does not do, you could with the following 
e = $<$4 6$>$,g=$<$2 1$>$,c=$<$4 1$>$
this would be distributes blocks of $<$4 6$>$ on 2 processors cyclicly,
to get 32 rows on 2 processors in blocks of 4 you need to cycle 4 times
hence c = $<$4 1$>$)
The shape of the distributed array could be thought of as being 
c cat g cat e.
(for the cyclic example the distributed shape would be $<$4 1 2 1 4 6$>$
which has the same number of components as $<$32 6$>$)
The compiler reduces all expressions to something like A=B but this
is a functional representation and does not account for the fact that
A and B are not on the same processor.  The operational assignement
can be done in three stages.  
\section*{The Algorithm}
\subsection*{Stage 1}
Stage one is to, for each processor, 
determine what sub-array of the assignement is done on that processor.
The assignment specifies a sub-array that is being assigned to.  Each
process owns (stores) a sub-array of the array begin assigned to.  The
geometric intersection of the sub-array being assigned to and the 
sub-array processor p owns is the part of the assignment that p needs to
perform.  For each processor, p, stage one computes the sub-array
of the array being assigned that must be done on processor p as well
as performing an intersection to determine what sub-array of the 
right hand side of the assignment is required for processor p. 
From now on let $<$2 4$>$:$<$4 8$>$ denote a sub-array that is equivalent
to ($<$4 8$>$-$<$2 4$>$+$<$1 1$>$) take ($<$2 4$>$ drop ARRAY)
Example: shape A and B=$<$12 30 10$>$ then to compute A=B on 4 processors,
processor 0 needs to assign:
$<$0 0 0$>$:$<$3 30 10$>$ of A=$<$0 0 0$>$:$<$3 30 10$>$ of B
processor 1 needs to assign:
$<$3 0 0$>$:$<$6 30 10$>$ of A=$<$3 0 0$>$:$<$6 30 10$>$ of B
processor 2 needs to assign:
$<$6 0 0$>$:$<$9 30 10$>$ of A=$<$6 0 0$>$:$<$9 30 10$>$ of B
processor 3 needs to assign:
$<$9 0 0$>$:$<$12 30 10$>$ of A=$<$9 0 0$>$:$<$12 30 10$>$ of B
End stage one.
\subsection*{Stage 2}
Stage two: Stage one contains a loop (for each p).  Stage two is performed
within that loop for each p.  So, the input to stage two is the sub-arrays
of the left hand side (lhs) and the (rhs) that need to be assigned on
the current p (processor).  Since the rhs may be distributed different
than the lhs, the rhs may not all be stored on processor p, even though
the lhs is.  The purpose of stage 2 is to gather up all the peices of
the rhs from different processor that are needed to perform p's portion
of the assignment.  Any particular sub-array of a distributed array
is contained on some sub-array of the processor array.  In stage
two there is a loop that loops through the sub-array of the processor
array.  Example: say the rhs sub-array for processor 1 is 
$<$3 0 0$>$:$<$9 30 10$>$ and the array is distributed by c=$<$1 1 1$>$,p=$<$4 1 1$>$,
e=$<$3 30 10$>$ then the sub-array of the processor array that contains
the above mentioned part of the rhs is $<$1 1 1$>$:$<$2 1 1$>$ of the processor
array (i.e. processors 1 and 2).
The loop s in the code does this loop from zero to the shape of this
processor sub-array, and sp is s plus the location of the processor
sub-array.  For the previous example $<$0 0 0$>$ $<$= s $<$ $<$2 1 1$>$ and
sp will be between $<$1 1 1$>$ and $<$2 1 1$>$ inclusive.
For each of the blocks an intersection is performed between the sub-array
needed for the rhs and the current block.  The result of this 
intersection is the sub-array of the rhs on processor s that is needed to
perform processor p's portion of the assignment.  This result is also
used to determine which part of the lhs is assigned this block that is
on processor s.
end stage two.
\subsection*{Stage 3}
stage three.  We can now perform the assignment.  At this point
we know what sub-array of the rhs on processor s should be assigned to
what sub-array of the lhs on processor p.  For stage three each processor
examines it processor number to see if it is p or s or nethier.   Lets
say I'm node i.  If i=p and i=s then p=s and I own both peices of the
lhs and rhs so I can do the assignment without any communication.  Else
if i=p then I own the lhs but need to get the rhs side from processor s
so I issue a read command.  Else if i=s then I own the rhs but not the
lhs so I need to send the rhs peice to processor p.
end stage three.
Whew!
Well, that was the original algorithm.  Of course I converted for
you from it's beautifully cryptic mathematical description 
but that would be hard to type here.  There is
only one difference between this and the current implementation and
that is the mode loop.  You want to make sure all the send are issued
before the reads so all three stages are performed twice.  The first
time through only sends are performed where needed and the second time
throught only receives and assignements are performed where needed.
\section*{Geometrical Overview}
The above algorithm requires some geometric intersections of 
n-dimensional rectangles.  Lets say we need the intersection of
rectangle A and B where each rectangle has two parts corresponding
to two opposite corners of the rectangle say c1 and c2.  There
are five cases for the intersection:
case 1 - A is completely inside B\\
B -------\\
  | --- |\\
  | |A| |\\
  | --- |\\
  -------\\
case 2 - B is completely inside A\\
A -------\\
  | --- |\\
  | |B| |\\
  | --- |\\
  -------\\
case 3 - A is above and left of B and intersect some\\
------  A\\
|    |\\
|  -------  B\\
|  |     |\\
---|     |\\
   -------\\
case 4 - B is above and left of A and intersect some\\
------  B\\
|    |\\
|  -------  A\\
|  |     |\\
---|     |\\
   -------\\
case 5 - A and B are disjoint\\
-------   -------\\
|     |   |     |\\
-------   -------\\
   A         B\\
The intersection code contains two if statements that are used to determine
which case to use.  Just before the if statements a bunch of temporary vectors
usually get created to compute the two corners of each rectangle (sub-array).
After the case is determined the intersection can be determined.
\end{document}
