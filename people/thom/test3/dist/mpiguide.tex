\documentstyle[margins]{article}

\title{Using MPI with the PSI Compiler}
\author{}
\date{}

\begin{document}
\maketitle

\section{Introduction}
The PSI Compiler is an array compiler that performs algebraic reductions on 
array expressions.  These reductions produce expressions that not only require
the least amount of temporary storage and execution time to compute, but also
attempt to provide the maximal utilization of cache.  The PSI Compiler also
has the ability to divide array expressions over many processors.  This is
done in a general manner so that the output code can be compiled and run on
any multiprocessor topology, including a network of workstations.  This
generality is enhanced through the use of a version of MPI developed at
Mississippi State University Engineering Research Center for Computational
Field Simulation.  MPI is a standard message passing interface that is
portable, scalable, and hopefully efficient.  Before integrating MPI into
the PSI Compiler, separate output source files were required for each
multiprocessor topology.  Now, with MPI, the same output source file can be
compiled on any multiprocessor topology that uses the MPI library.  

What follows are directions on how to use MPI with the PSI Compiler. 
A more general discussion of the PSI Compiler can be found in the ``PSI
Compiler User's Guide'', which describes the compiler and MOAL grammar
in further detail.

\section{The MOAL program}
The MOAL program needs only two lines at the beginning of the source file
(in addition to the source code) in order to be able to use MPI.  They are:
{\samepage
\begin{verbatim}
$arch mpimac
$procs N
\end{verbatim}
}
where N is the number of processors to be used.  You should not specify a
hostname with the \$hostname directive since MPI does not use a host.  Here
is an example MOAL program:
{\samepage
\begin{verbatim}
$arch mpimac
$procs 4
test1 (array in1^3 <20 30 10>, array in2^3 <20 30 10>, array out1^3 <20 30 10>)
{
    printout (in1);
    printout (in2);
    out1 = in1 + in2;
    printout (out1);
}
\end{verbatim}
}
\section{The Main Program}
The main program is the program that should initialize data and call the
functions generated by the compiler.  Here is an example main program for
the function test1 that will be generated by the above MOAL program.
{\samepage
\begin{verbatim}
#include <stdio.h>

printout(double what[])
{
  int i, j;

  for (i=0; i<5; i++) {
    for (j=0; j<300; j++) 
      printf ("%4lg ",what[i*300+j]);
    printf ("\n");
  }
}

main (int argc, char *argv[])
{
  int i, j, k, l;
  double in1[20*30*10], in2[20*30*10], out1[20*30*10];

  l=0;
  for (i=0; i<20; i++) {
    for (j=0; j<30; j++) {
      for (k=0; k<10; k++) {
        in1[i*300+j*10+k]=(double)(l);
        in2[i*300+j*10+k]=(double)(l);
        l++;
      }
    }
  }

  test1(argc, argv, in1, in2, out1);
}
\end{verbatim}
}
\section{The Makefile}
Your Makefile should be similar to the following one.  node.m is the MOAL
source file, node.c is the output of the MOA compiler, node\_main.c is
the main program you need write in order to call the functions contained in
the C output of the the MOA compiler, and node is the actual executable.  
{\samepage
\begin{verbatim}
CC= cc
MC= mc
COPTIONS = 
CLIBNAMES = -lmoautil -lnsl -lsocket -lthread -lmpi -lm
MFLAGS=

# Use these for Sun
CINCLUDEPATHS = -I/usr/local/psi/include -I/software/all/mpi/include
CLIBPATHS = -L/usr/local/psi/lib/sun -L/software/all/mpi/lib/solaris/ch_p4

CFLAGS = $(COPTIONS) $(CINCLUDEPATHS)
CLIBS = $(CLIBPATHS) $(CLIBNAMES)

MCOUTS = node.c 
MCOBJS = node.o 
OBJS   = node_main.o 
EXES   = node

all:   $(EXES)

node: node.o node_main.o
	$(CC) $(CFLAGS) -o node node.o node_main.o $(CLIBS)

node_main.o: node_main.c
	$(CC) $(CFLAGS) -c node_main.c

node.o: node.c
	$(CC) $(CFLAGS) -c node.c

node.c : node.m
	$(MC) node.m -o node.c

clean:
	rm -f $(MCOUTS) $(EXES) $(OBJS) $(MCOBJS)
\end{verbatim}
}

\section{Execution}
Execution of the program is done through use of the script mpirun, which can
be found in the directory /software/all/mpi/sun on UMR's Sun Sparc 10
workstations. Running the executable program in parallel can be done in the
following way
\begin{verbatim}
mpi_run -mr_np N program
\end{verbatim}
where N is the number of processors you want to use (this should be the
same number of processors specified with the \$procs directive of the MOAL
program) and progarm is the executable program.  Running the executable
created by the above makefile can be done with the following command line:
\begin{verbatim}
mpi_run -mrnp 4 node
\end{verbatim}
\end{document}
