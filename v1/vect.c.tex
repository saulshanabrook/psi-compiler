force_vect
  vect_t *force_vect (vect_t *vect)
    vect:  the vector to be assigned to a temporary variable
    returns:  a new vector representing the temporary variable
  This function creates a temporary variable for simple variable
  access and assigns vect to this new variable in the C output.  The new
  variable has a name of the form tmp_vectn, where n is a number from
  0 to a very large value.  This function is like purify_vect, except
  it ALWAYS assigns vect to a temporary variable.

make_vect
  vect_t *make_vect (expr_t *expr)

print_scalar
  print_scalar (vect_t *expr)

purify_vect
  vect_t *purify_vect (vect_t *vect)
    vect:  the vector to be purified of elements containing expressions
    returns:  a simple vector
  This function checks to see if vect is a simple vector, and if so, it
  is returned.  Otherwise, a temporary vector is created, and vect is
  assigned to this vector in the C output.  

rav_value
  double rav_value (vect_t *vect, int i, int *simple)
    vect:  the vector whose ith value will be returned
    i:  the index of the value to be returned
    simple:  TRUE if the (i+index)th value of vect can be evaluated at
             compile time;  FALSE otherwise
    returns:  the floating point value of the (i+index)th element of
              vect
  This function attempts to evaluate the value of the ith element
  of the vector vect.  This value can be evaluated if 1) the vector
  is of type CONSTANT, 2) the vector is of type IOTA, 3) the vector
  is of type CONST_ARRAY, 4) the vector is of type RAV and the
  (i+index)th value is a SCALAR (index is the number representing
  the starting index of the vector), or 5) the vector is of type
  EMBEDDED_ARRAY.  Otherwise, it is not evaluated, and simple
  remains FALSE, the value it was given at the beginning of the
  function.

simplify_vect
  vect_t *simplify_vect (vect_t *vect)
    vect:  the vector to be simplified
    returns:  a simplified vector
  This function searches through the vector, combining all the 
  constants and constant expressions.

static_shps
  int static_shps (vect_t *vect)
    vect:  the vector in question
    returns:  TRUE or FALSE
  This function determines if vect->index, vect->loc, and vect->shp are
  scalars, and if the vector is just a variable access.

tau
  void tau (FILE *fd, expr_t *expr)
    fd:  the file to write the output to 
    expr:  the expression to find the tau of
    returns:  nothing
  This procedure outputs an expression representing the value of \tau (expr).
  This expression may be a single value, or a series of constants
  (representing the extents of each dimension) multiplied together.  
  Output is to the file fd.

vect_assign
  vect_assign (vect_t *vect, vect_t *res, char *op)

vect_drop
  vect_t *vect_drop (vect_t *left, vect_t *right)

vect_cat
  vect_t *vect_cat (vect_t *left, vect_t *right)

vect_len
  vect_t *vect_len (vect_t *vect)
    vect:  the vector whose length is to be found
    returns:  a vect_t type representing the length of vect
  This function will return zero_vec if vect is NULL, and if vect
  is not NULL, will return the length of the vector (technically
  the shape).

vect_op
  vect_t *vect_op (int op, vect_t *left, vect_t *right)

red_rav
  s_expr_t *red_rav (vect_t *vect, int i)

vect_setup
  int vect_setup (vect_t *expr)

vect_take
  vect_t *vect_take (vect_t *left, vect_t *right)
 
vect2array
  vect_t *vect2array (char *name,vect_t *vect)
    name:  the name of the new array (can be NULL)
    vect:  the vector to be turned into an array
    returns:  a new vector that represents the new array
  This function creates a new vector, gives it a name, and calls
  vect_assign to assign vect to this new vector in the C output.  
  Any element of vect that is an expression is evaluated in the C
  output and stored in the corresponding element of the new vector.
  This allows future C statements to access the values of vect 
  through a simple variable access to array name.

vect_unop
  vect_t *vect_unop (int op, vect_t *vect)

