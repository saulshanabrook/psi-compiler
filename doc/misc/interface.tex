\documentstyle[margins]{article}

\title{MOA Compiler Interface}
\author{}

\begin{document}
\maketitle

\section{Introduction}

  The MOA compiler is split up into several modules each responsible for
a particular group of tasks.  In order to use your parser and lexical analyzer
with the MOA compiler you will need to interface with these modules.  The 
following is a brief description of all the modules and types you will need.  

\begin{tabular}{lcl}
psi.h &	: & Contains the types used for representing MOA expressions. \\
 & & These include ident\_t, s\_expr\_t, vect\_t, expr\_t, and parser\_t \\
ops.h &	: & Defines a type that represents complex operators. \\
sys.o &	: & This contains system related procedures. \\
ident.o	& : & All functions that manipulate ident\_t and s\_expr\_t objects. \\
vect.o & : & All functions that manipulate vect\_t objects. \\
moa\_func.o & : & Contains functions for building MOA expressions and  \\
 & & performing shape analysis. \\
psi.o &	: & All the routines for reducing array expressions and \\
 & & performing code generation for expressions.
\end{tabular}

  You will need to call functions from the sys, moa\_func, and psi modules.
Each module has a .e file associated with it.  This file contains external
definitions for that module, and you will need to include it, with the
include directive, in any C file you write that calls a function from that
module.  Your parser will also need to include psi.h and ops.h in order to
use those types.

\section{Writing the Parser}
 
  You should write a main C file that will call all the initialization 
procedures, open output file, and call your parser to start parsing the program.
The output file to be opened should to be the file pointer cfile that has
already been declared in the sys module.  The following initialization
functions need to be called in this order before starting to parse the 
input file.
\begin{verbatim}
  mem_init();
  psi_init();
\end{verbatim}

\subsection{Handling Procedure Declarations}
  Next you will construct a parser that interfaces with the MOA compiler.
For each procedure in the input file that you parse, you should print a C
procedure header for that procedure to the output file.  You should also 
initialize the following variables to zero that have been declared in other 
modules.

\begin{verbatim}
 max_dim=0;
 forall_num=0;
 red_num=0;
 temp_array_num=0;
 const_array_num=0;
 vect_tmp_num=0;
 scalar_num=0;
 temp_used=FALSE;
\end{verbatim}

  You then need to open a temporary files vfile and tfile that are already 
declared in the sys module.  The code generation will print variable 
declarations to the vfile and code to the tfile.  Also print C variable 
declarations for every array declaration you parse to the vfile.  If there is a
declaration of an array with an unknown size you will print the appropriate 
malloc statement to the tfile.  You will also need to allocate and initialize 
an ident\_t structure for each declaration in the input file.  Next build a MOA
expression tree by calling the functions in the moa\_func module as you parse 
expressions.  When you parse the end of the procedure you should close both
vfile and tfile.  The following code then needs to be performed to allocated
a couple of variables that are used during code generation.

\begin{verbatim}
  for (i=0; i<max_dim; i++) {
    fprintf(cfile,"  int i%d;\n",i);
  }
  fprintf(cfile,"  int step1[%d];\n",max_dim);
  fprintf(cfile,"  int step2[%d];\n",max_dim);
\end{verbatim}

Now copy vfile and tfile to the output file.

\subsection{Handling The Procedure Body}

  There are only two statements that can appear in the procedure body, a
forall statement and an assigment statement.  When parsing the forall statement
you will need to call 
start\_loop with loop\_variable, left\_bound, right\_bound as arguments and
save the return value in a variable.  All of the input variables should be 
pointers to expr\_t type objects as shown in psi.e.  The return value is an
integer.  The parser can now parse the body of the forall loop and at the end
of the loop body should call end\_loop(depth) where depth is the return value
you saved from start\_loop.  For the assignment statement you can simply
parse the variable on the left hand side and the expression on the right hand
side.  This will give you two things of type expr\_t, the result and the
expression.  After each assignment you should call as
psi\_assign(expression,result) and this will reduced and generated code for the
assignment.

\subsection{Building the MOA Expressions}

  You will need to use the moa\_func module in order to build MOA expressions
of expr\_t type.  All of the functions in moa\_func.c accept and return
parameters of type parser\_t except the psi\_access function.  When you
parse a declaration of an array you should have created an object of ident\_t
type for it.  When an variable is referenced in an expression you can call
psi\_access with that ident\_t object and it will return a parser\_t object that
you can use in building the expression.  The functions in moa\_func need to be
called in posfix order so that when any operator is encountered you already 
have a parser\_t object for the expressions that are the left and right
arguments.  Now calling the moa\_func with these arguments will return a 
parser\_t for a new expression that is the arguments combined with the operator.

  There is also an extra argument to the psi\_omega2 function that is type
op\_t.  The structure describes a complex operator that is created by the 
omega function.  It is basically a linked list of omega operators with an
non-omega function at the tail.

\end{document}
