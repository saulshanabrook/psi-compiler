\documentstyle[epsf]{article}

\addtolength{\topmargin}{-.75in}
\addtolength{\textwidth}{1.6in}
\setlength{\textheight}{8.75in}
\addtolength{\oddsidemargin}{-0.75in}
\addtolength{\evensidemargin}{-0.75in}
\setlength{\parskip}{.1in}

\newcommand{\cat}{+\!\!\!\!+}
\newcommand{\take}{\,\bigtriangleup\,}
\newcommand{\drop}{\,\bigtriangledown\,}
\newcommand{\karate}{\!\widehat{\hbox to 10pt{}}\!}
\newcommand{\rshp}{\widehat{\rho}}

\title{PSI Compiler\\Guide to the dist.c and part.c}
\author{}

\begin{document}
\maketitle
\tableofcontents

\section{Overview}

\subsection{part.c}
This is the partitioner module.  The top level procedure is $partition$
and accepts a statement list as an argument.  The partitioner calls
$rec\_partition$ and $resolve\_rules$.  $rec\_partition$ searches the
assignment statements for all arrays that are used.  If an array
is used and has an explicit rule then that rule is added to a global
list.  If the array does not have an explicit rule, a default rule
is created that partitions the array evenly over the first dimension.

When the $rec\_partition$ returns, a global list of all the rules for
all the arrays is stored in the $rules$ variable.  $resolve\_rules$
is then called.  This procedure will eventually resolve any conflicting
rules and create a final distribution list.  The procedure currently assumes
there are no conflicts (and in fact there can't be yet) and just converts
them to the final distribution form (the dist\_t type).  The list
of final distributions is returned from the $partion$ procedure and
will be passed to the code generation module.

\subsection{dist.c}
The distribution module contains an initialization procedure $dist\_init$
that must be called when the compiler is invoked.  The two main procedures
in this module are $init\_dist\_arrays$, and $code\_dist$.  
$init\_dist\_arrays$ is called
by $code\_c$ before any code generation is performed.  This procedure will
generate code to distribute initial data over the processors.  The second
procedure, $code\_dist$ generates code for a list of expressions, involving
distributed arrays, that can be combined.  Currently, no expressions involving
distributed arrays may be combined so the list must be a list of one element.
The procedure works in two phases.  First $dist\_lhs$ is called.  This 
procedure
creates a loop that loops through all processors, say $p$, and at each 
iteration determines what part, say part1, of the array on the left hand side 
is
owned by the processor $p$.  Then the procedure computes what part of the right
hand side is required for the assignment to part1.  Call this part2.  Now
$dist\_rhs$ is called.  This procedure loops through all the processor numbers,
say $sp$ and determines what part, say part3, of part2 belongs to processor
$sp$.  The expression can now be coded.  If $p=sp=$my\_processor then
I own part1 of the left hand side and part3 of the right hand side, so I can
compute the expression (between part1 and part3).  Otherwise if 
$p=$my\_processor then I own part1 but 
not part3 so I should wait to receive something from processor $sp$ and then 
use that to compute the expression.  Otherwise if $sp=$my\_processor then
I own part3 but not part1 so I should send part3 to processor $p$.  Otherwise
I do not own part1 or part3 and can continue without doing anything.

\end{document}
