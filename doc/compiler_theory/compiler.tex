\documentstyle[margins]{article}
\newcommand{\cat}{+\!\!\!\!+}
\newcommand{\take}{\,\bigtriangleup\,}
\newcommand{\drop}{\,\bigtriangledown\,}
\newcommand{\karate}{\!\widehat{\hbox to 10pt{}}\!}
\newcommand{\rshp}{\widehat{\rho}}
\newcommand{\one}{<\! 1\!>}
\newtheorem{definition}{Definition}
\newcommand{\Size}{\tau\,}
\newcommand{\Dim}{\delta\,}
\newcommand{\Ravel}{\,\mbox{\tt rav}\,}
\newcommand{\Reshape}{\,\widehat{\rho}\,}
\newcommand{\Reduce}{\,\mbox{\tt red}\,}
 
 
\title{A Reduction Semantics for Array Expressions:\\The PSI Compiler
\thanks{This work was supported by the National Science
          Foundation's Presidential Faculty Fellow Award Program}}
\author{Lenore Mullin \and Scott Thibault
\thanks{L. Mullin, {\tt lenore@cs.umr.edu}, 
and S. Thibault, {\tt sthib@cs.umr.edu}, Department of
     Computer Science, University of Missouri-Rolla, Rolla, Missouri 65401
USA}}
\begin{document}
\maketitle
\section*{Introduction} 
High Performance Computing is not only concerned with the performance of
a program on a very fast processor, but also high performance
on many fast processors perhaps of different speeds and topologies. 
Scientific computation and modeling often require real time
visualizations on high resolution graphics terminals. The 
data structure most widely used in scientific computation is the
array. Arithmetics operations and permutations are often used
where its associated algebra transcends many
scientific disciplines. In order to achieve high performance
computing, the interfaces with which people interact with computers
must be rapid and accurate with the ability to move across
many platforms and adapt quickly to change. If a programming
language supports an algebra useful to scientists, the programming
language, and subsequent compiler, must optimize the resources of
a particular machine or machines. That is, the compiled
program must allocate the least amount of memory and do the least
amount of arithmetic. Ideally, we'd like to minimize
latency and bottlnecking and achieve load balancing of all used
resources. These issues may be addressed by employing
The Psi Calculus\cite{mul88,mdst92,mdst93,mdst94,tm93} 
as a backend compiler to scientific languages. 
The reduction rules of the Psi Calculus, which
describe how to simplify array expressions,
were recently automated in the Psi
Compiler\cite{tm93}. This paper presents the results of this effort.


\section*{Shape Analysis}
 
In order to perform reductions on array expressions,  the shapes of all 
variables and partial results must be known.  The only 
given shapes are those of the variables and are acquired while parsing the 
array declarations.  The shape of each partial result is computed in terms
of these given shapes.  This process occurs from the innermost of the expression
outward (a postfix order).  We will see in the next section that the process 
of reduction occurs from the outermost inward (an infix order)
\footnote{Order is irrelevant since the $\psi$-calculus has the Church-Rosser
Property.}.  
So these
are conceded distinct processes and require that the entire expression tree
be retained.
 
Given that we have a complete expression tree, we can compute the shape of 
every node in
the tree.  The shape at a given node in the tree represents the resulting 
shape of the expression represented by the sub-tree rooted in the given
node.  The leaves of the tree will be variable accesses and the shape of them
will be know from the variable declarations.  Visiting the nodes in a postfix 
manner ensures that when a node is visited that the shapes of it's arguments
(children) have already been computed.
The shape of the internal nodes is computed in terms of the shapes
of it's children by the following rules\cite{mul88}:
\footnote{The Appendix presents a brief introduction to a Mathematics
of Arrays(MOA) and the psi calculus.}
\begin{center}
Shape Calculation
\end{center}
\begin{itemize}
\item
Dimension: $\rho(\delta\xi) \equiv \Theta$
\item
Shape: $\rho(\rho\xi) \equiv <\delta\xi>$
\item
Psi: $\rho(\vec{v}\psi\xi_{r}) \equiv  (\tau\vec{v})\drop (\rho\xi_{r})$
\item
Concatenation: $\rho(\xi_{l}\cat\xi_{r}) \equiv  ((1\take (\rho\xi_{l}))+(1\take (\rho\xi_{r})))
\cat (1\drop (\rho\xi_{l}))$
\item
Algebraic operators: $\rho(\xi_{l}\;\hbox{op}\;\xi_{r}) \equiv  \rho\xi_{l}
\equiv \rho\xi_{r}$
\item
Scalar extension: $\rho(\sigma\;\hbox{op}\;\xi) \equiv  \rho\xi$
\item
Iota (scalar): $\rho(\iota\sigma) \equiv  <\!\sigma\! >$
\item
Iota (vector): $\rho(\iota\vec{v}) \equiv \vec{v}\cat (\rho\vec{v})$
\item
Take: $\rho(\sigma\take\xi) \equiv  <\! |\sigma |\! >\cat (1\drop (\rho\xi))$
\item
Drop: $\rho(\sigma\drop\xi) \equiv ((1\take (\rho\xi))-|\sigma |)
\cat (1\drop (\rho\xi))$
\item
Omega: $\rho(\xi_{l}\: _{\mbox{op}}\Omega_{<\sigma_{l} \sigma_{r}>} \xi_{r})
	\equiv \vec{u}\cat\vec{v}\cat\vec{x}\cat\vec{w}$
 
where
\begin{eqnarray*}
m & \equiv &  \mbox{min }(((\delta\xi_{l})-\sigma_{l}),((\delta\xi_{r})-\sigma_{r})) \\
\vec{u} & \equiv & (-m)\drop((-\sigma_{l})\drop\rho\xi_{l}) \\
\vec{v} & \equiv & (-m)\drop((-\sigma_{r})\drop\rho\xi_{r}) \\
\vec{x} & \equiv & (-m)\take((-\sigma_{l})\drop\rho\xi_{l}) \\
\vec{w} & \equiv & \rho(((\vec{i}\cat\vec{k})\psi\xi_{l})\mbox{op}
	((\vec{j}\cat\vec{k})\psi\xi_{r})) \\
 & & \mbox{for } \;\;0\leq^{*}\vec{i}<^{*}\vec{u},\\
& & 0\leq^{*}\vec{j}<^{*}\vec{v}, \;\;
	0\leq^{*}\vec{k}<^{*}\vec{x} \\
\end{eqnarray*}
\end{itemize}
 
Having these rules we can compute the shape of every node in the tree.  When
each node has an expression for it's shape, shape analysis is complete and we
can continue with the reduction of the expression.
 
\section*{Reduction}
 
All  array operators (excluding the algebraic operators) are 
available to allow the designer to shift things around and access sub-arrays
of an array.  If an assignment statement involves no algebraic operators and
only array operators then no data is changed. It is only relocated in the 
resulting array.  Since the data is only being relocated, no temporary 
variables should be required for execution of the assignment.  In all previous
work temporary variables are used for each array operation during the 
execution of an expression.  The process of reduction involves reducing an 
array expression from one that uses
complex array operators to one that only involves memory accesses.  Thus we
present a method of reduction to generate an efficient implementation of
an array expression.
 
As an example lets look at the assignment of a MOA expression to an array.
$$A=B\cat(<\! 1\; 2\! > \drop C)$$ where $\rho B=<\! 2\; 4\! >$, $\rho 
C=<\! 4\; 6\! >$
and $A$ is conformable.  It's easy to see that we can realize this assignment 
without any temporaries by the following expressions.
$$\begin{array}{llll}
\vec{i}\psi A & = & \vec{i}\psi B &;\forall <\! 0 \; 0\! >\leq^{*}\vec{i}
<^{*}<\! 2\; 4\! > \\
(\vec{i}+<\! 2\; 0\! >)\psi A & = & (\vec{i}+<\! 1\; 2\! >)\psi C &
;\forall <\! 0\; 0\! >\leq^{*}\vec{i}<^{*}<\! 3\; 4\! >
\end{array}$$
We would ideally like the
compiler to be able to generate these two expressions and avoid the use of
the temporary variables.  This is in fact possible.
 
All array expressions in MOA have the property that they can be reduced
to a normal form.
The resulting normal form of an expression involves only memory accesses
and algebraic operators.  Furthermore, deriving the normal form of an 
expression is a mechanical process that is easily implemented on 
computational machines.
So the normal form is indeed what we need the 
compiler to find, and it can be achieved mechanically.
 
The goal for the compiler will be to input an expression and reduce it into
separate expressions (one for each variable in the expression) with the
array operators eliminated.  The input expression is in the form
$A=B$ where $B$ is an expression.
We want each output statement to be in the
form
$$\forall 0\leq^{*}\vec{v}<^{*}\vec{b}, \;\;(\vec{v}+\vec{l})\psi A=
(\vec{v}+\vec{i})\psi B$$
In this expression $\vec{b}$ refers to the bound, $\vec{l}$ refers to the
location of the assignment in the complete result (since the expressions 
are reduced to separate expression, each reduced expression represents only
one part of the result), $\vec{i}$ is the top left-hand corner of the
sub-array that is to be assigned and $B$ is an array variable.  
If this is the output we desire and
we're given an input expression $A=\xi$ then we'll start the reduction process
by assuming that the reduced expression is just
$$\forall 0\leq^{*}\vec{v}<^{*}\vec{b}, (\vec{v}+\vec{l})\psi A=
(\vec{v}+\vec{i})\psi \xi$$ where $\vec{b}\equiv\rho\xi, \vec{l}\equiv^{*}0$,
and $\vec{i}\equiv^{*}0$.  This expression is correct and is
in the right form.  However, it may not be the reduced, normal form.  If the 
top node of the expression tree for $\xi$ is a variable access then this is 
the normal form and we're done.  If the top node is an operator then the
expression needs more work to find the reduced expression(s).  In this
case we need to eliminate the top operator by using a rewrite rule,
creating one or two new expressions.
 
This process is probably best seen by an example.  The example given earlier
was $$A=B\cat(<\! 1\; 2\! > \drop C)$$ $\rho B=<\! 2\; 4\! >$, $\rho 
C=<\! 4\; 6\! >$
and $A$ is conformable.  Our starting assumption is 
$$\forall 0\leq^{*}\vec{v}<^{*}<\! 5\; 4\!>, (\vec{v}+<\! 0\; 0\!>)\psi A=
(\vec{v}+<\! 0\; 0\!>)\psi (B\cat(<\! 1\; 2\! >\drop C))$$. 
This expression is not free from array operators
so it's not in normal form.  We notice though that the expression can be
rewritten, eliminating the $\cat$, as
$$\begin{array}{llll}
\forall 0\leq^{*}\vec{v}<^{*}<\! 2\; 4\!>, & (\vec{v}+<\! 0\; 0\!>)\psi A 
& = & (\vec{v}+<\! 0\; 0\!>)\psi B  \\
\forall 0\leq^{*}\vec{v}<^{*}<\! 3\; 4\!>, & (\vec{v}+<\! 2\; 0\;>)\psi A 
& = & (\vec{v}+<\! 0\; 0\!>)\psi (<\! 1\; 2\! >\drop C)
\end{array}$$
The first expression here is in normal form since it contains no array 
operators (simple $\psi$ operations are only for memory access).  The 
second expression is not in normal form so we continue reducing that
expression.  That can be done by noticing that we can rewrite this expression,
eliminating the $\drop$, as
$$\forall 0\leq^{*}\vec{v}<^{*}<\! 3\; 4\!>, (\vec{v}+<\! 2\; 0\!>)\psi A  = 
(\vec{v}+<\! 1\; 2\!>)\psi C$$
This expression is in normal form so now the reduction is complete and the
final result is the two expressions
$$\begin{array}{llll}
\forall 0\leq^{*}\vec{v}<^{*}<\! 2\; 4\!>,& (\vec{v}+<\! 0\; 0\!>)\psi A 
& = & (\vec{v}+<\! 0\; 0\!>)\psi B  \\
\forall 0\leq^{*}\vec{v}<^{*}<\! 3\; 4\!>,& (\vec{v}+<\! 2\; 0\!>)\psi A 
& = & (\vec{v}+<\! 1\; 2\!>)\psi C
\end{array}$$
These are equivalent to the expressions given in the first example at the
beginning of this section.
 
The only mystery in this process is rewriting the expression in order to 
eliminate the outermost operator (the top node of the sub-tree).  This
elimination can be systematically done based on a set of rules for each
operator.  Each operator is eliminated such that one or two expressions are
produced.
Each rule for an operator consists of an expression for computing
the new bound, location, and indexes, for the one or two new expressions.  
The expressions in each rule are dependent only on the shapes of the arguments
and the function represented by that rule.  The shapes are already known from
shape analysis.  The rewrite rules that the compiler uses during the reduction
are given in the next section.
 
Thus we have a deterministic way to perform the reduction of an array 
expression based on our previous results from shape analysis.  We start
with the assumption that the reduced expression is 
$$\forall 0\leq^{*}\vec{v}<^{*}\vec{b}, (\vec{v}+\vec{l})\psi A=
(\vec{v}+\vec{i})\psi \xi$$ where $\vec{b}\equiv\rho\xi, \vec{l}\equiv^{*}0$,
$\vec{i}\equiv^{*}0$, $A$ is the result and $\xi$ is the expression.  
We now examine the outermost operator, if it is
not a simple variable access we apply the rule for that operator generating
one or two new expressions, otherwise we are done.  If we generate new
expressions then we examine the outermost operator of these expressions and
follow the same process in a recursive manner until no new expressions are
generated.  When this process is complete, we have a set of expressions 
that represent the reduced form of the original expression.
 
\section*{Reduction Rules}
\begin{figure}
\framebox[6.5in][l]{\parbox{6.2in}{\obeylines
program := \{ procedure\_definition \} \\
procedure\_definition := PROCEDURE\_name "(" formal\_parameter\_list ")" block\_body \\
formal\_parameter\_list := parameter\_definition \{ "," parameter\_definition \} \\
parameter\_definition := "int" PARAMETER\_name \verb+|+
	"array" PARAMETER\_name array\_definition \\
array\_definition := "$\karate$" INTEGER\_number "$<$" \{ INTEGER\_number \verb+|+ 
	INTEGER\_PARAMETER\_name \} "$>$" \\
block\_body := "\{" definition\_part statement\_list "\}"
definition\_part := \{ constant\_definition\_part \verb+|+ variable\_definition\_part \verb+|+
	global\_definition \} \\
constant\_definition\_part := constant\_definition ";" \verb+|+ 
	constant\_scalar\_definition ";" \\
constant\_definition := "const" "array" VARIABLE\_name array\_definition "="
	vector\_constant \\
constant\_scalar\_definition := "const" "array" VARIABLE\_name "$\karate$0 $<>$=" number \\
vector\_constant := "$<$" \{ number \} "$>$" \\
variable\_definition\_part := variable\_definition ";" \\
variable\_definition := array VARIABLE\_name array\_definition \\
global\_definition := "global" VARIABL\_name ";" \\
statement\_list := \{ statement ";" \} \\
statement := assignment\_statement \verb+|+ for\_statement \verb+|+ allocate\_statement \\
allocate\_statement := "allocate" identifier variable\_access \\
for\_statement := for "(" term "$<$=" variable\_access "$<$" term ")" "\{" 
	statement\_list "\}" \\
assignment\_statement := variable\_access "=" expression \\
variable\_access := VARIABLE\_name \verb+|+ PARAMETER\_name \\
expression := factor \{ operator expression \} \\
operator := "+" \verb+|+ "-" \verb+|+ "*" \verb+|+ "/" \verb+|+ "psi" \verb+|+ "take" \verb+|+ "drop" \verb+|+ "cat" \verb+|+ "pdrop" \verb+|+
	"ptake" \verb+|+ operator "omega" constant\_vector \\
unary\_operator := "iota" \verb+|+ "dim" \verb+|+ "shp" \verb+|+ "+ red" \verb+|+ "- red" \verb+|+ "* red" \verb+|+ 
	"/ red" \verb+|+ "tau" \verb+|+ "rav" \\
factor := term  \verb+|+ "(" expression ")" \verb+|+ unary\_operator factor \\
term := variable\_access \verb+|+ constant\_vector \\
variable\_access := identifier; \\
identifier := letter \{ letter \verb+|+ digit \verb+|+ '\_' \} \\
constant\_vector := "$<$" \{ number \} "$>$" \\
}}
\caption{The MOA compiler input language specification}
\label{bnf}
\end{figure}
The compiler is built using standard techniques, plus the added function
of performing shape analysis and the reduction.  The shape analysis is
performed as the input program is being parsed.  The shape of an expression
is calculated from the shapes of it's arguments by the definitions listed
in the previous section.  When an expression has been completely parsed
it is reduced to a normal form by successively eliminating the outermost
operator in the expression.  The elimination is performed based on the
definition of the operator.  The rules used for the functions in our
grammar (Figure \ref{bnf}) follow.

In the last section the general form that expressions were reduced to assumed
that the dimensions of the arrays were equal.  Thus the length of $\vec{v}$
would be the same as the length of $\vec{l}$, and $\vec{i}$.  However
in general this won't be true so instead the following general form is
used.
$$\forall 0\leq^{*}\vec{v}<^{*}\vec{b}, \;\;(\vec{v}+\vec{l})\psi(\vec{m}\psi 
        A)=(\vec{v}+\vec{i})\psi(\vec{n}\psi \xi)$$
In this form $\vec{l}\cat\vec{m}$ is a full index representing the location
in $A$ and $\vec{i}\cat\vec{n}$ is a full  index representing the location
in $\xi$.  This allows $\tau\vec{v}$ to be smaller than the dimension of either
array.


 
\begin{itemize}
\item{Dim}
$$\forall 0\leq^{*}\vec{v}<^{*}\vec{b}, \;\;(\vec{v}+\vec{l})\psi(\vec{m}\psi 
	A)=(\vec{v}+\vec{i})\psi(\vec{n}\psi (\delta\xi))\Rightarrow$$
$$
\forall 0\leq^{*}\vec{v}<^{*}\vec{b}, \;\;(\vec{v}+\vec{l})\psi(\vec{m}\psi A)=
(\vec{v}+\vec{i})\psi(\vec{n}\psi \xi.dim)$$
 
\item{Shape}
$$\forall 0\leq^{*}\vec{v}<^{*}\vec{b}, \;\;(\vec{v}+\vec{l})\psi(\vec{m}\psi
	A)=(\vec{v}+\vec{i})\psi(\vec{n}\psi (\rho\xi))\Rightarrow$$
$$\forall 0\leq^{*}\vec{v}<^{*}\vec{b}, \;\;(\vec{v}+\vec{l})\psi(\vec{m} \psi
	A)=(\vec{v}+\vec{i})\psi(\vec{n}\psi \xi.shape)$$
 
\item{Psi}
$$\forall 0\leq^{*}\vec{v}<^{*}\vec{b}, \;\;(\vec{v}+\vec{l})\psi(\vec{m}\psi 
	A)=(\vec{v}+\vec{i})\psi(\vec{n}\psi(\vec{j}\psi\xi))\Rightarrow$$
$$\forall 0\leq^{*}\vec{v}<^{*}\vec{b}', \;\;(\vec{v}+\vec{l}')\psi(\vec{m}'
	\psi A)=(\vec{v}+\vec{i}')\psi(\vec{n}'\psi\xi)$$
where $\vec{b}', \vec{l}', \vec{m}', \vec{i}',$ and $\vec{n}'$ are defined by
\begin{eqnarray*}
\vec{b}' & \equiv & \vec{b} \\
\vec{l}' & \equiv & \vec{l} \\
\vec{m}' & \equiv & \vec{m} \\
\vec{i}' & \equiv & \vec{i} \\
\vec{n}' & \equiv & \vec{j}\cat\vec{n}
\end{eqnarray*}
 
\item{Take}
$$\forall 0\leq^{*}\vec{v}<^{*}\vec{b}, \;\;(\vec{v}+\vec{l})\psi(\vec{m}\psi 
	A)=(\vec{v}+\vec{i})\psi(\vec{n}\psi(\vec{j}\take\xi))\Rightarrow$$
$$\forall 0\leq^{*}\vec{v}<^{*}\vec{b}', \;\;(\vec{v}+\vec{l}')\psi(\vec{m}'
	\psi A)=(\vec{v}+\vec{i}')\psi(\vec{n}'\psi\xi)$$
where $\vec{b}', \vec{l}', \vec{m}', \vec{i}',$ and $\vec{n}'$ are defined by
\begin{eqnarray*}
\vec{b}' & \equiv & \vec{b} \\
\vec{l}' & \equiv & \vec{l} \\
\vec{m}' & \equiv & \vec{m} \\
\vec{x} & \equiv & ((\vec{j}<0)(\vec{j}+((\tau\vec{j})\take\rho\xi))+
	(\tau\vec{j}\take\vec{i}))\cat ((\tau\vec{j})\drop\vec{i}) \\
\vec{i}' & \equiv & (\tau n)\drop\vec{x} \\
\vec{n}' & \equiv & (\tau n)\take\vec{x}
\end{eqnarray*}
 
\item{Drop}
$$\forall 0\leq^{*}\vec{v}<^{*}\vec{b}, \;\;(\vec{v}+\vec{l})\psi(\vec{m}\psi 
	A)=(\vec{v}+\vec{i})\psi(\vec{n}\psi(\vec{j}\drop\xi))\Rightarrow$$
$$\forall 0\leq^{*}\vec{v}<^{*}\vec{b}', \;\;(\vec{v}+\vec{l}')\psi(\vec{m}'
	\psi A)=(\vec{v}+\vec{i}')\psi(\vec{n}'\psi\xi)$$
where $\vec{b}', \vec{l}', \vec{m}', \vec{i}',$ and $\vec{n}'$ are defined by
\begin{eqnarray*}
\vec{b}' & \equiv & \vec{b} \\
\vec{l}' & \equiv & \vec{l} \\
\vec{m}' & \equiv & \vec{m} \\
\vec{x} & \equiv & ((\vec{j}>0)(\vec{j})+((\tau\vec{j})\take\vec{i}))\cat
	((\tau\vec{j})\drop\vec{i}) \\
\vec{i}' & \equiv & (\tau n)\drop\vec{x} \\
\vec{n}' & \equiv & (\tau n)\take\vec{x}
\end{eqnarray*}
 
\item{Cat}
$$\forall 0\leq^{*}\vec{v}<^{*}\vec{b}, \;\;(\vec{v}+\vec{l})\psi(\vec{m}\psi 
	A)=(\vec{v}+\vec{i})\psi(\vec{n}\psi(\xi_{l}\cat\xi_{r}))\Rightarrow$$
$$\forall 0\leq^{*}\vec{v_{l}}<^{*}\vec{b_{l}}, \;\;(\vec{v_{l}}+\vec{l_{l}})
	\psi(\vec{m_{l}}\psi A)=
	(\vec{v_{l}}+\vec{i_{l}})\psi(\vec{n_{l}}\psi \xi_{l}),$$
$$\forall 0\leq^{*}\vec{v_{r}}<^{*}\vec{b_{r}}, \;\;(\vec{v_{r}}+\vec{l_{r}})
	\psi(\vec{m_{r}}\psi A)=
	(\vec{v_{r}}+\vec{i_{r}})\psi(\vec{n_{r}}\psi \xi_{r})$$
where $\vec{b_{l}}, \vec{l_{l}}, \vec{m_{l}}', \vec{i_{l}}, \vec{n_{l}}, 
	\vec{b_{r}}, \vec{l_{r}} \vec{m_{r}}, \vec{i_{r}},$ and 
	$\vec{n_{r}}$ are defined by
\begin{eqnarray*}
\vec{b'} & \equiv & [( 1\take(((\rho\vec{n})\rshp 1)\cat\vec{b}))\lfloor
	(( 1\take\rho\xi_{l})-( 1\take(\vec{n}\cat\vec{i})))]
	\cat( 1\drop(((\rho\vec{n})\rshp 1)\cat\vec{b})) \\
\vec{b_{l}} & \equiv & ((\vec{b}'[0]>0)(((\tau\vec{n})\drop\vec{b}')[0]))\cat
	((\tau\vec{n}+1)\drop\vec{b}') \\
\vec{l_{l}} & \equiv & \vec{l} \\
\vec{m_{l}} & \equiv & \vec{m} \\
\vec{x_{l}} & \equiv & (( 1\take\rho\xi_{l})\lfloor( 1\take
	(\vec{n}\cat\vec{i})))\cat
	( 1\drop(\vec{n}\cat\vec{i})) \\
\vec{i_{l}} & \equiv & (\tau n)\drop\vec{x_{l}} \\
\vec{n_{l}} & \equiv & (\tau n)\take\vec{x_{l}} \\
\vec{b_{r}} & \equiv & (( 1\take\vec{b})-( 1\take\vec{b_{l}}))\cat
	( 1\drop\vec{b}) \\
\vec{l_{r}} & \equiv & (( 1\take\vec{l})+( 1\take\vec{b_{l}}))\cat
	( 1\drop\vec{l}) \\
\vec{m_{r}} & \equiv & \vec{m} \\
\vec{x_{r}} & \equiv & (( 1\take(\vec{n}\cat\vec{i}))-
	( 1\take\vec{x_{l}}))\cat( 1\drop(\vec{n}\cat\vec{i})) \\
\vec{i_{r}} & \equiv & (\tau n)\drop\vec{x_{r}} \\
\vec{n_{r}} & \equiv & (\tau n)\take\vec{x_{r}} \\
\end{eqnarray*}
 
\item{Omega (binary)}
$$\forall 0\leq^{*}\vec{v}<^{*}\vec{b}, \;\;(\vec{v}+\vec{l})\psi(\vec{m}\psi 
	A)=(\vec{v}+\vec{i})\psi(\vec{n}\psi
	(\xi_{l}\: _{\mbox{op}}\Omega_{<\;\sigma_{l} \sigma_{r}\;>}
	\xi_{r}))\Rightarrow$$
Forall $(h\take(\vec{n}\cat\vec{i}))\leq^{*}\vec{j}<^{*}
	(h\take(\vec{b_{0}}+(\vec{n}\cat\vec{i})))$
$$\forall 0\leq^{*}\vec{v}<^{*}\vec{b}', \;\;(\vec{v}+\vec{l}')\psi(\vec{m}'
	\psi A)=(\vec{v}-\vec{i}')\psi(\vec{n}'\psi 
	[((((\delta\xi_{l})-\sigma_{l})\take\vec{j})\psi\xi_{l}) \mbox{op}
	((((\delta\xi_{r})-\sigma_{r})\take\vec{j})\psi\xi_{r})]) $$
where $m, h, d_{1}, \vec{b_{0}}, \vec{b}', \vec{l}', \vec{m}', \vec{i}',$ and 
	$\vec{n}'$ are defined by
\begin{eqnarray*}
m & = & ((\delta\xi_{l})-\sigma_{l})\lfloor ((\delta\xi_{r})-\sigma_{r}) \\
h & \equiv & ((\delta\xi_{r})-\sigma_{r}+(\delta\xi_{l})-\sigma_{l}-m \\
d_{1} & = & \delta [\xi_{l}\: _{\mbox{op}}\Omega _{<\;\sigma_{l} \sigma_{r}\;>}
	\xi_{r}] \\
\vec{b_{0}} & \equiv & ((d1-(\tau\vec{b}))\rshp 1)\cat\vec{b} \\
\vec{b}' & \equiv & h\drop\vec{b_{0}} \\
\vec{l}' & \equiv & h\drop(\vec{m}\cat\vec{l}) \\
\vec{m}' & \equiv & (h\take(\vec{m}\cat\vec{l}))-(h\take(\vec{n}\cat\vec{i})) \\
\vec{i}' & \equiv & h\drop(\vec{n}\cat\vec{i}) \\
\vec{n}' & \equiv & \Theta \\
\end{eqnarray*}
\end{itemize}
 
\section*{The Compiler Implementation}
\begin{figure}                  
\framebox[6.5in][l]{\parbox{6.2in}{
\begin{tabbing}
\obeylines
test(array A$\karate$2 $<\! 5\; 4\!>$, array C$\karate$2 $<\! 4\; 6\!>$)\\
\\
\{ \= \\
\>  const array B$\karate$2 $<\! 2\; 4\!>=<\! 1\; 2\; 3\; 
	4\; 1\; 2\; 3\; 4\! >$;\\
\\
\>  A=B cat ($<\! 1\; 2\!>$ drop C);\\
\} \\
\end{tabbing}
}}
\caption{An example input file for the MOA compiler}
\label{ex_moa}
\end{figure}
\begin{figure}[h]
\framebox[6.5in][l]{\parbox{6.2in}{
\begin{tabbing} 
\obeylines
\#include \verb+<+stdlib.h\verb+>+\\
\#include "moalib.e"\\
test(float \_A[], float \_C[])\\
 
\{ \= \\
\>  int \= i0;\\
\>  int i1;\\
\>  int step1[2];\\
\>  int step2[2];\\
\>  int shift,offset;\\
\>  float \_B[]=\{1.000000, 2.000000, 3.000000, 4.000000, 
	1.000000, 2.000000, 3.000000, 4.000000\};\\
\>  shift=0*4+0;\\
\>  offset=0*4+0;\\
\>  for (i0=0; i0<2; i0++) \{ \\
\>\>    for \= (i1=0; i1<4; i1++) \{ \\
\>\>\>      *(\_A+shift)=*(\_B+offset); \\
\>\>\>      shift++; \\
\>\>\>      offset++;\\
\>\>    \}\\
\>  \}\\
\>  shift=2*4+0;\\
\>  offset=1*6+2;\\
\>  step1[0]=2;\\
\>  for (i0=0; i0<3; i0++) \{\\
\>\>    for (i1=0; i1<4; i1++) \{\\
\>\>\>      *(\_A+shift)=*(\_C+offset);\\
\>\>\>      shift++;\\
\>\>\>      offset++;\\
\>\>    \}\\
\>\>    offset+=step1[0];\\
\>  \}\\
\}\\
\end{tabbing}
}}
\caption{The output of the MOA compiler for the example}
\label{ex_C}
\end{figure}
A preliminary compiler has been implemented to realize these reduction 
techniques.  The compiler was written in the C language and can be
used on several platforms.  These platforms include networks of Sun, Next,
and SGI workstations, and the Connection Machine CM-5.
The compiler excepts an input program of
array expressions written in MOA notation.  The compiler compiles the 
input program directly into C output.  The input language is described in
figure \ref{bnf}.  Notice that all that remains after the psi reductions is
the manipulation of address pointers for B and C.
 
The procedure definition provides an interface to C programs by allowing
integers and pointers to arrays to be passed as parameters to the MOA 
procedure.
 
Now we can write a procedure {\it test} to implement the expressions we used
in the reduction section.  The input program for our example is shown in
figure \ref{ex_moa}.  The resulting C output from compiling the $test$ 
procedure is shown in figure \ref{ex_C}.


\section*{Conclusion}
Our decomposition and mapping technologies to one or more processors also
employs the Psi Calculus \cite{mdst92,mdst93,mdst94}.  
The techniques employed in 
\cite{mdst93} were extended and scaled to a CM-5, a topic of our most
recent  paper\cite{mdst94}.  
Our recent efforts are to interface scientific languages such as
Fortran 90 and to benchmark our performance with other Fortran 90 compilers.
We are also decomposing problems using PVM.  We will determine if our
automatic techniques compare to PVM's performance and versatility.  Future
work is to support heterogeneous processing and load balancing.
 
\clearpage

\bibliographystyle{plain}
\bibliography{compiler}

\clearpage
\appendix

\section*{Appendix A: Introduction to the Psi Calculus}

The definitions that follow are all that is needed to follow the
proofs within this paper. For full definitions and typing
see\cite{mul88,hai91}.
Angle brackets $<\;>$ and square brackets $[\;]$ are used to help visualize
the dimensionality of an array. For example, 3 is a scalar hence
no brackets are used. If we wanted to denote a one component
vector we'd say $<3>$, a one row one column matrix by $[3]$, etc.
Multiple brackets show dimensionality, i.e. $[[2\;3\;4]]$ illustrates
a 1 by 1 by 3 array. We call this information shape or form and denote
it by $<1\;1\;3>$. Scalars are denoted by lower case alphabetics or $\sigma$,
vectors by lower case alphabetics with an arrow above and higher
dimensional arrays by capital letters or $\xi$. $<\;>$ and $\Theta$ denote
the empty vector. A superscript  denotes the dimensionality.
All objects in the $\psi$ calculus are arrays, e.g. scalars are
0-dimensional arrays, vectors are 1-dimensional arrays, matrices
are 2-dimensional arrays, etc..
\begin{definition}
The unary prefix operation $\delta$ returns the dimensionality
of an array.
\[ \delta A^{n} \equiv n\].
\end{definition}
\begin{definition}
An array, $A$ has a shape, denoted by $\rho A$, which is a
vector whose entries are the lengths of each of A's dimensions.
\[ \rho A \equiv < s_{0}, \ldots,  s_{(\delta A)-1}> \]
where the $s_j$'s are non-negative integers, $0 \leq j < (\delta A)-1$
\end{definition}
\begin{definition}
The total number of components of an array, $\tau A$, is defined whenever
\[ \tau A \equiv \pi ( \rho A )  \equiv \prod_{i=0}^{(\delta A)-1} (\rho A)_{i}
\]
\end{definition}
\begin{definition}
Concatenation of two vectors $\vec x \cat \vec y$ is defined whenever
\[
\rho \vec x \cat \vec y \equiv < (\tau \vec x ) + (\tau \vec y)>\]
and $\forall i$, s.t. $0 \leq i < ((\tau \vec x ) + (\tau \vec y))$
\[
(\vec x \cat \vec y)[i] \equiv \left\{
\begin{array}{ll}
\vec x [i] & \mbox{if $ 0 \leq i < (\tau \vec x )$} \\
\vec y [i - (\tau \vec x)] &
\mbox{if $ (\tau \vec x) \leq i < ((\tau \vec x) + (\tau \vec y)) $ }
\end{array}
\right. \]
\end{definition}
\begin{definition}
Mapping valid index vectors to lexicographic coordinates is done using
$\gamma$. If we assume  $\tau \vec i \equiv \tau \vec s \equiv j$.
$\gamma$ is defined whenever
\begin{eqnarray*}
\gamma(<\;>\;;\;<\;>) &\equiv &0 \\
\gamma(<i>\;;\;<s>) &\equiv& i \\
\gamma(<i_{0} \cdots i_{j-1}>\;;\;<s_{0} \cdots s_{j-1}>) & \equiv &
i_{j-1} + ( s_{j-1} \times \gamma ( <i_{0} \cdots i_{j-2}>\;;\;< s_{0} \cdots
s_{j-2}>))
\end{eqnarray*}
noting that $0 \leq ^{*} \vec i <^{*} (\rho A) \Rightarrow
0 \leq  \gamma( \vec i\;;\;(\rho A ) < ( \tau A)$.
\end{definition}
\begin{definition}
The reshape operator $\Reshape$ restructures an array
having the same contents to a new shape or {\em form}.
Given a vector $\vec v$ containing non-negative integers,
$\vec v \Reshape A $ is well defined providing
$\pi \vec v \equiv \pi <\rho A >$.
\end{definition}
\begin{definition}
The $\Ravel$ operator flattens an array into a vector having the
same content s.t. $\rho \Ravel A \equiv <\pi (\rho A)>$.
\end{definition}
\begin{definition} Take $\bigtriangleup$ and drop $\bigtriangledown$ are
needed to define the indexing function $\psi$. Given $n$ a non negative
integer s.t. $ 0 \leq n < ( \tau \vec v )$
\begin{eqnarray*}
\rho (n \bigtriangleup \vec v )\equiv <n> \\ \rho n (\bigtriangledown \vec v )
\equiv < ( \tau \vec v ) - n >
\end{eqnarray*}
and $\forall i,j$ s.t. $ 0 \leq i < n$ and $0 \leq j < (\tau \vec v)-n$
\begin{eqnarray*}
( n \bigtriangleup \vec v ) [i] \equiv \vec v [i] \\
( n \bigtriangledown \vec v) [i] \equiv \vec v [n+1]
\end{eqnarray*}
\end{definition}
We define the $\psi$ function to
take any valid index of an array and return a scalar or subsection
of an array, e.g. a row or plane.
We showed in \cite{hai91} how $\psi$ was defined inductively on its
first argument as follows:
\begin{definition}
The indexing function $\psi$ is defined whenever
\begin{itemize}
\item $\Theta \psi A \equiv A$
\item Given: $\vec i$ is a valid index vector,
$\vec s $ is  an array shape and
$\vec c$  denotes the components of an array with shape $\vec s$.
\[\psi \equiv \lambda \vec i \lambda \vec s \lambda \vec c.\; {\bf if}\;[\tau (\vec i)\ = 0, \;\vec s \Reshape \vec c\;, \psi ( 1 \bigtriangledown 
\vec i,
1 \bigtriangledown \vec s, ( \pi 1 \bigtriangledown \vec s )
\bigtriangleup ( \vec i [0] \times ( \pi 1 \bigtriangledown \vec s ))
\bigtriangledown \vec c )] \]
\item The shape or {\em form} of $\vec i \psi A$, is defined
whenever
\[ \rho (\vec i \psi A) \equiv (\tau \vec i ) \bigtriangledown (\rho A ) \]
\end{itemize}
\end{definition}
\begin{definition} Index generation on scalars or vectors is defined
whenever
\begin{eqnarray*}
\rho (\iota n) \equiv <n> &\;& \rho (\iota \vec v) \equiv
\vec v \cat (\rho \vec v)
\end{eqnarray*}
and $\forall i$ s.t. $0 \leq i < n$ and
$\forall \vec i$ s.t. $0 \leq^{*} \vec i <^{*} \vec v$
\begin{eqnarray*}
(\iota n) [i] \equiv i &\;&
\vec i \psi  (\iota \vec v) \equiv \vec i
\end{eqnarray*}
\end{definition}
\begin{definition} $\xi_{l} R^{*} \xi_{r}$ is defined whenever $R$ is a scalar 
relation, and $\rho\xi_{l}\equiv\rho\xi_{r}$.  $\xi_{l} R^{*} \xi_{r}$ is
true iff $$\forall\vec{i}\;\mbox{s.t.}\; 0\leq^{*}\vec{i}<^{*}\rho\xi_{l},\;\;
(\vec{i}\psi\xi_{l}) R (\vec{i}\psi\xi_{r})$$ is true.  Furthermore if
$\sigma$ is a scalar, we write $\sigma R^{*} \xi$ to mean, for all valid
$\vec{i},\;\;\sigma R(\vec{i}\psi\xi)$.  This is also equivalent to
$\xi R^{*}\sigma$.
\end{definition}
\begin{definition}
$a\lfloor b\equiv \mbox{min}(a,b)$
\end{definition}

\end{document}
