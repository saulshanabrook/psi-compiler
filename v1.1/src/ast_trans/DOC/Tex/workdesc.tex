Description of my Work

What I Did Not Do

When asked what I am doing for my Masters Project, I (and others)
often use the phrase {\it providing an interface to the PSI compiler}.
To use that as the title for this writeup would be misleading.  In
December 1995, I was given the parameters of my assignment.  The
input:  an HPF AST.  The output:  an HPF AST.  The side effects:  PSI
reductions on the original AST.  Even with these sparse specifications,
it is clear that an interface would be desirable.  Separation of
concerns, extensibility, modularity, portability; each of these
doctrines would be addressed with such a development.  However, several
factors contributed to this approach not being taken:\\

\begin{itemize}
\item  My inexperience with compilers.  It would require most of my efforts to
comprehend the two compilers involved in the project; to design part of yet
another would have resulted in a substantial increase in time and work.\\

\item  Approaching deadlines.  Even if I had sufficient background to design
such an interface, to do so comprehensively would have taken time away
from the more crucial task of feeding programs to and from the endpoints.
I suspect that if benchmarks had been completed midway through the Spring
semester as planned, I would have become involved in the design of the
actual interface that is necessary for future research.\\

\item  My inexperience with programming languages.  Ideally, we would like
the interface to the PSI compiler to be extensible to any programming
language that can express array operations.  It is hard to say at this
point whether I could have designed an interface that would port to any
other language, or could support other array features.  As I mention
later in this writeup, my original approach had to be abandoned because
I was only considering a handful of functions that I would have to
support.\\

\item  My incomplete knowledge concerning the PSI Calculus reduction rules.
There is no question that any intermediate representation I might have
chosen would have more greatly resembled the procedural, brute-force
array operations of Fortran 77, rather than the elegant, recursive
interpretations found in MOA.  Certainly, we want this representation
to "sit close" to the reduction engine, for purposes of portability
and extensibility.\\
\end{itemize}


What I Did Do

My primary goal was to convert pghpf ASTs to an equivalent Psi expression,
and then to convert the Psi statement lists to equivalent pghpf statements.
I used stacks on both sides to maintain information hiding (although I
breached that once or twice in order to save time).  I made my approach
consistent enough so that my modules could be expanded by other students
as is deemed necessary.  In this paper, I don't simply describe the final
product; the amount of code currently being used in my modules is relatively
small (hundreds of lines), and the project is currently still being worked on.
Rather, I discuss the dead-ends I've reached and the backtracking I've done in
order to handle the awkward output of the Psi compiler, and also some of the
issues one should consider when designing the true interface to the Psi
compiler.
