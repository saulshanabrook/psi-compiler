Overview of Compilers

\begin{itemize} 
\item Organization of the Project \\ \\

This project consist of four segments.  The first segment traverses abstract
syntax trees (ASTs) produced by {\em pghpf }  and passes relevant information to
functions provided by the second segment, which creates equivalent abstract
syntax trees that the Psi compiler can recognize.  The third segment
traverses a data structure of Psi "statements" and passes relevant information
to functions provided by the fourth segment, which creates equivalent
abstract syntax trees that are inserted into the program being processed
by the first segment.  \\ \\

The introduction of a true interface would cause this program to be broken
up into five logical segments.  The first segment would convert ASTs of the
target compiler into ASTs of this interface.  The second segment would
convert ASTs of the interface into ASTs of the Psi compiler.  The third
segment would convert ASTs of the Psi compiler to ASTs of the interface.  The
fourth segment would convert ASTs of the interface to ASTs of the target
compiler.  Finally, a fifth segment would consist of several modules needed
to support the interface, including functions to create ASTs, and symbol
table implementation and routines.\\ \\

\item PGI Compiler\\ \\

  The goal of the project was to demonstrate the Psi compiler's effectiveness
  in simplifying array expressions used in traditional benchmarks.  We
  obtained access to the source code for {\em pghpf } , the High Performance
  Fortran (HPF) compiler produced by The Portland Group, Inc. (PGI)\\ \\

  The top level function for my project is called PGI_PSI_AST.  We inserted
  a call to this function in the pghpf source code immediately after an
  HPF program has been parsed and put into AST form, but before any
  optimization of the program takes place by pghpf.  This is because the
  output of the fourth segment of my program could potentially create
  several loops that could be merged by pghpf's optimizer.\\ \\

  The first segment of my project traverses a PGI statement list and
  removes assignment statements from this list.  It then verifies
  that this statement is one that can be converted to an equivalent
  Psi expression that the Psi compiler's reduction engine can handle.
  For example, at this time it appears that the Psi compiler won't
  accept an assignment statement where the LHS is an array section.
  PGI statements are not nested.  For example, a node for a DO statement
  does not contain a sublist corresponding to the body of the loop.
  Rather, it just contains the loop index information.  The statement
  list will contain a ENDDO statement after the loop body.  This allowed
  me to process any assignment statement in the program without having
  to encode this code segment using recursion, particularly in the form of
  statement list stacks.  In the future, one desirable
  feature of the interface would be to analyze this statement list and
  look for reductions that could be applied to several lines.  For
  example, consider the following code segment:\\ \\

  TEMP1 = CSHIFT(B, shift = 5)\newline  
  TEMP2 = CSHIFT(C, shift = 3)\newline 
  A = TEMP1 + TEMP2\\ \\

  As is, my project wouldn't eliminate the use of the temporary arrays.
  If it could inspect the statement list and verify that the specified
  values of TEMP1 and TEMP2 weren't used as R-values elsewhere in the program,
  it could eliminate these three statements and then replace this segment
  with the reduced code for:\\ \\

  A = CSHIFT(B, shift = 5) + CSHIFT(C, shift = 3)\\ \\

  This would allow programmers to not have to "bunch up" expressions
  into a single statement.  For now, we have had to do this with the
  benchmarks.  In summary, my function that processes single PGI
  assignment statements (PGI_CONV_AST) has one PGI statement as its
  input, and produces a PGI statement list as its output.\\ \\

\item  PGI Representation\\ \\

  In this representation, the nodes of the statement list exist primarily
  to implement a doubly linked list.  Each node of a list points to an AST node.
  Thus, a statement is treated like an expression.  Each AST node consists
  of 7 32-bit words.  Information is then packed into this node via C
  macros.  To accommodate the wide variety types of ASTs, a particular
  piece of the node may be used to represent different fields based on
  the ASTs type.  For example, the section used to hold the optype of
  an arithmetic expression is also used to hold the initial expression
  for the index variable of a do loop.\\ \\

  The symbol table is represented in a similar fashion, with each symbol
  table entry consisting of 10 32-bit words.  Symbol table routines are
  also implemented with C macros.  There are also similar, but distinct,
  structures for function argument lists, triple lists (e.g., for forall
  statements), and array subscript information.  Despite the semantic
  differences between these objects, every one of them is referred to by
  a 16-bit, unsigned integer.  Although the use of well-named macros made
  programming for this compiler very easy, with the data structure
  information well hidden, the homogeny of the representation made tracing
  errors a tedious process, as they usually compiled.\\ \\

  \item Psi Compiler and Psi Representation\\ \\

  The Psi Compiler was written by former students of Dr. Mullin.  The
  core of the compiler was written without a front end {\it per se } .
  However, students wrote a parser that could handle a subset of the
  Mathematics of Arrays (MOA).  The core {\em does }  incorporate a
  back end, however, and it is this feature that has been the primary
  impediment to my progress.  The code generation is in C, and the
  programmers of the module that implements the reduction rules of the Psi
  Calculus clearly had this end in mind, as I will discuss in length.\\ \\

  The input to the reduction engine of the Psi compiler is an assignment
  statement.  However, the LHS can only take a particular form, and
  reductions are done on the RHS expression.  The output is a statement
  list that corresponds to the procedural steps that would have to be
  taken in order to complete the assignment.  There is a research group
  currently discussing whether and how the reduction engine should be
  rewritten.  One of the primary goals is to keep expressions as
  expressions, much as the Psi Calculus does.  In this approach, the
  output of reducing the RHS would be the equivalent expression in
  {\bf normal form } .  This would have the advantages of:
    \begin{itemize} 
    \item  aiding in proving the correctness of the reduction engine;
    \item  easing the addition of reduction rules to the engine;
    \item  allowing for a uniform processing of the output.
    \end{itemize} 
  \\ \\
  The Psi compiler utilizes numerous data structures.  The {\bf expr_t } 
  type is used to represent array expressions.  It consists of many
  fields, but not all fields are used for all types of expressions.  For
  forall expressions, the index, bound, and stride fields are used to
  describe the index variable's behavior.  For array access expressions,
  these fields describe how to access the array, often in terms of
  a forall index variable.  In addition, larger expressions can be formed
  using the op, left, and right fields.  All expressions use the shape
  field.\\ \\

  The {\bf ident_t }  type is generally used to represent a symbol table
  entry.  It contains fields for the name, type and value of an
  identifier.  It is also used for what I feel are some unrelated
  representations.  For example, the contents of a vector are stored
  in an ident_t rather than a vect_t.  The ident_t type has a field
  {\em array }  which is a struct of fields that are never all used
  at the same time.  It allows the contents of a vector to be stored
  in either an array of doubles, or an array of scalar expressions.
  I never understood why this distinction needed to have been made
  when a scalar expression {\em can }  be a double.  Also, the
  designers have allowed one to build identifier expressions, but
  only with strict left-to-right evaluation.  It seems to me that this
  is a counterintuitive and undesirable feature.\\ \\

  The {\bf vect_t }  type allows one to construct vector expressions without
  using the bulky expr_t structure.  The {\bf s_expr_t }  type allows
  one to build scalar expressions.  This too allows one to avoid using the
  expr_t type.  The existence of these types may have made it easier for
  the original programmers to create ASTs, but it makes traversing
  the results much more difficult, since just about every expression
  contains at some level each of these types.  In addition, while I
  have no problem with several types having a shape component, it was
  somewhat confusing that the type of the shape depends on the type
  of the structure in which it appears!  expr_t's have a vect_t shape,
  vect_t's have an s_expr_t shape, and the shape of the ident_t's array
  field is an array of ident_t's.  When you consider that shapes have 
  shapes themselves, you realize just how confusing this approach is.
  I would hope that the future designers of this compiler try to
  homogenize these representations to some degree, or at least use C++ and
  create overloaded shape constructor functions.\\ \\

  Finally, the {\bf statement_t }  type is used as an intermediate
  representation for code generation.  That is, the reduction engine
  not only applies rules to the incoming expression, but also translates
  that abstract syntax tree into another language.  This type is used
  to implement a linked list, with each node containing a pointer to
  a node of some type depending on the type of statement.  Usually, the
  type of a statement is {\bf reduced_t } .  This field will often have
  a subfield (of type {\bf assign_t } ) which represents an assignment
  similar to the one that was passed to the engine.  However, 
  more than one may be assignment returned.  Additionally, reduced_t's
  can be used to implement forall loops, in which case you will find
  inside this node a list of reduced_t's which represent the body of
  the loop.\\ \\

  \item  PSI_PGI / PGI_PSI stacks\\ \\

  To allow communication between the two compilers, I implemented
  a pair of stacks which are manipulated by {\em driver }  routines.
  I have tried my best to maintain the hiding of information.  Most
  parameter passing occurs at the atomic level (constants and identifiers.)
  Larger expressions obtain the data for subexpressions/operands/arguments
  from the stack.
  Occasionally, I felt that it was more efficient (and conceptually
  clearer) to pass constants over.  One such context was the dimension
  for array operations.  However, as the Psi compiler expands and becomes
  more comprehensive, this will have to be changed.  For example, the
  shift amount of a circular-shift (cshift) should be any scalar expression.
  While the design of the Psi data structures allows for this expression,
  the compiler rejects this usage.  This is surprising, since I have found
  that even the current back end of the Psi compiler allows for arbitrary
  scalar expressions within vectors, and assumes that reduction created
  scalar expression trees, even for constants.  Anyways, I have
  written my PSI_PGI code to accommodate this anticipated change.\\ \\

  The PGI_PSI stack nodes are of type {\bf parser_t } , which is essentially
  an ident_t and an expr_t joined together.  I feel that the parser module
  is the best developed of all the modules of the Psi compiler; there are
  many functions available for the construction of parser_t's from other
  data types (including identifiers and vectors).  If the rest of the
  compiler were written as diligently as the one that creates parser_t's,
  programming for this software would be an easier task.\\ \\

  The PSI_PGI stack nodes are of type UINT16 as defined in pghpf.  Recall
  that this type is used to represent all PGI objects.  It is really
  used as an index of a lookup table; the macros "select" the correct
  table.  I use them to pass atomic objects, array indexing, and all
  types of expressions (scalar, vector, and array).\\ \\ 
\end{itemize} 

