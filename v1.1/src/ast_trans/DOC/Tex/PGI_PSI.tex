Converting from pghpf to Psi \\ \\

Segment I: PGI to PGI_PSI Stack \\ \\

The first segment of my project involves traversing a PGI representation
of an HPF program and passing relevant information to the PGI_PSI stack.
At the top most level, I visit each node of a linked list of statements.
If the node represents an assignment statement, the LHS is visited, followed
by the RHS.  I then request the PGI_PSI stack to form an assignment
statement and pass it to the reduction engine.  To enforce modularity,
I pass a pointer to the current statement node to the PGI_PSI modules.  These
modules will, in turn, pass them back to the PSI_PGI modules when requesting
to add a new statement to the PGI list.\\ \\

Most of my PGI_PSI functions are straightforward.  I traverse the PGI
expressions depth-first, so combinators such as binary operators or
function calls are handled by first visiting the sub-expressions (and
pushing the translations onto the PGI_PSI stack), and then requesting
the PGI_PSI stack to rebuild the expression.  Unlike statements, I
traverse expressions right-to-left as a rule.  For procedure/function
calls, I process the last argument first.  Generally, the PGI_PSI routines
are "ignorant" of Psi notation.  For example, when a CSHIFT node is
encountered, a request is made to PGI_PSI_CSHIFT; the PGI_PSI module
will pop the argument off the stack and form an equivalent Psi expression
using take and drop.\\ \\

  \begin{itemize}
  \item  Array Identifiers\\ \\
  
  In MOA, all arrays have first index of 0.  In HPF, arrays may have any
  starting index.  So, for each dimension of the array, I compute the
  number of components (stop - start + 1) and store the results in an
  array.  I pass this array to the PGI_PSI stack, which treats it as
  the shape vector for the array.\\ \\

  \item  Spread combined with Binop\\ \\

  This is a feature that I started to implement, but abandoned since it
  wasn't the top priority.  Since I originally was going to build the
  SPREAD on the Psi side, I wanted to be efficient in the case that the
  SPREAD was being used with a binary operator.  There are certain conditions
  where it is more efficient to use the original array as an operand and
  then to spread the result.  However, I am now considering handling
  SPREAD much like subscripting; rather than concatenating copies of
  the array together, I would instead create a forall node (with shape to
  reflect the size of the SPREAD), and then express indexing the SPREAD in
  terms of indexing the original array (which is how my Fortran 90 manual
  actually defines SPREAD).

  \item  Array Subscripting\\ \\

  I shifted the responsibility of how to implement array sections in MOA
  to the PGI_PSI stack.  Subscript nodes point to an array of triples which
  describe the start/stop/strides for each dimension.  Note that in HPF,
  these values are absolute indices, so a triple of (5:9:2) refers to
  {\em indices }  5, 7, and 9, rather than elements 5, 7, and 9.  Since
  the array information passed over set the starting index to 0, I need
  to adjust the triples before pushing them onto the stack.  Since arrays
  can only have indexing of stride 1, I need only to subtract the starting
  index of the subscripted array from the start and stop of the triple.
  So, if the above triple were being used to subscript an array with
  starting index of 3, the triple would become (2:6:2). \\ \\

  \item  Expression Subscripting\\ \\

  Normand Belanger extended the pghpf grammar to allow a programmer
  to specify expression subscripting.  The syntax is rather strict:
  the expression to be subscripted {\em must }  be enclosed within
  parentheses (e.g. (CSHIFT(A, 5))(5:9)).  I didn't have to add anything
  to my module to accommodate for this; the parser will screen out what
  can or cannot be subscripted.  However, I should point out the hazards
  of using this feature.  Suppose in the above expression, A is a
  one-dimensional array with indices 5 through 15.  The expression "A"
  in pghpf has shape <11>, and has indices 5 through 15.  The expression
  (CSHIFT(A,5)) also has
  shape <11>, but has indices 1 through 11.  That is, when an array
  is directly subscripted, then the triples specify the indices of
  the array to be taken.  However, when an arbitrary array is subscripted,
  the triples specify the elements of the resulting array to be taken. \\ \\
  \end{itemize}  

  Presently, the reduction engine of the Psi compiler can only process
  a subset of legitimate expressions.  For example, only integer constants
  can be used as a shift amount.  Therefore, as I traverse the PGI ASTs,
  I verify that the expression is Psi-ready before I pass the assignment
  to psi_reduce.  Instead of making a first pass to inspect the expression,
  I make a single pass, assuming that it will be valid.  If I find at some
  point that the current statement should be skipped, I abort the traversal
  immediately and empty the PGI_PSI stack. 

  Segment II: PGI_PSI Stack to Psi\\ \\

  The second segment of my program is a set of routines that remove
  information from the PGI_PSI stack and form Psi expression trees.
  Though this seems straightforward, it actually took me a fair amount
  of time to write these functions in such a way that the reduction
  engine excepted the new ASTs.  I did not find routines provided that
  would, say, take a constant and return a parser_t that represented
  that constant.  There was, however, a function ({\em psi_access } )
  that accepted an ident_t and built a parser_t around it.  So, for
  the atomic types, I merely had to experiment with filling in the
  fields of the ident_t type.  Before the first statement is converted,
  a symbol table is created and initialized; this table is used by
  all statements throughout an entire program.  Scope may present a
  problem later on as this module becomes more ambitious; this may need to be
  addressed, both in the symbol table routines and in my modules.\\ \\

  Most of my implementations of the intrinsic functions are variations
  of routines found in hpfm_func.c, a module written for the HPF front-end
  to the Psi compiler.  The moa_func.c module of the Psi compiler
  will form parser_t's for cat, take, drop, unary, and binary operators.
  Additionally, it provides routines that build omega expressions based
  on the input (two arguments, operation, and partition).  {\bf Note } :
  due to discrepancies between HPF and MOA, my routines subtract one
  from the dimension before forming MOA expressions.\\ \\

  {\bf Example } :  CSHIFT(A, shift = 5, dim = 3)\\ \\

  Segment I will place A on the stack, and then call PGI_PSI_CSHIFT(5, 2).
  PGI_PSI_CSHIFT will retrieve A from the stack, store the constant 5
  into a parser_t, decrement the dimension, and then call:\\
  \begin{enumerate}
  \item psi_omega2(5, A, [drop, 1, 2]);  (the '2' refers to dim)
  \\
  \item psi_omega2(5, A, [take, 1, 2]); (the '2' refers to dim)
  \\
  \item pgi_omega2((1.), (2.), [cat, 2, 2]); (both '2's refer to dim)
  \end{enumerate} \\ \\

  Subscripting - A First Approach

  When one is forming Psi expressions, there are really two approaches to
  take.  The first is to use basic Psi operators to try to "build" the
  result; the approach in PGI_PSI_CSHIFT uses this approach, as does most
  of the intrinsic-conversion routines.  The second approach is to indicate
  how indexing this expression relates to indexing the operands.  This is
  a difficult concept to relate, and, unfortunately, I was stuck in the former
  mode until very recently.  I'll use 2-D TRANSPOSE as an example.\\ \\

  To "build" a 2-D TRANSPOSE of MxN array A, one would have to build an
  expression using M*(N-1) + (M-1) omega (concat) operators.  That is, we
  isolate the elements within the row and then use omega to concatenate
  them into a column, which would require N-1 concats for each of the M rows.
  Then, we would join the M columns together to form an NXM transpose of A.\\ \\

  More efficient is to express indexing the TRANSPOSE in terms of indexing A.
  One can build two forall nodes (both with start <0>, one with bound <M>, the
  other with bound <N>), say with forall indices {\em forall1 }  and
  {\em forall2 } .  Then the bottommost node of this expression would
  be an array access (of A) with index vector <forall2 forall1>.  The AST
  is much smaller than in the previous approach, and there is virtually
  nothing left to reduce.\\ \\

  I first tried to "build" the array section by forming an index set
  (iota of the shape of the target section), multiplying the index set
  by the stride of the section, adding in the start of the section, and
  then using omega to index A with the index set.  Unfortunately, the Psi
  compiler produced complex code for this type of expression.\\ \\

  I then realized how much easier it would be to express subscripting
  not just by using a new forall node, but by actually changing the
  start/stop/stride information in the expression to be subscripted.
  The first step was is compute the new shape of the node; the number of
  components in each dimension {\em i }  would be (stop[i] - start[i])
  / stride[i]) + 1.  The second step is to compute the new indexing
  rule for this node; if the node has no prior indexing rule, then it
  is initialized to (start - 1), the decrement being due to the fact that
  HPF starts indexing at 1 by default, whereas all Psi indexing starts at 0.
  If the node does have a prior indexing rule, 
  then a new start must be computed.  Recall that if the start of the
  section is 5, that refers to the fifth element of the expression.  So,
  the adjusted start is the original start + (section start - 1) * original
  stride.
  The stride is set; if there was a stride originally specified, then the
  new stride is the original stride multiplied by the section stride.\\ \\

  {\bf Example } : Let {\em fish }  be a vector with shape <50>, and {\em 
  goat }  be a vector with shape <4>.
  The statement (fish(10:40:3))(3:9:2) would be evaluated as follows:
  \begin{enumerate}
  \item The process is bottom-up, so first the array access node is
  created with un-set index, bound, and stride fields.
  \item Next, the section (10:40:3) is processed.
  \item The array access node's index field is set to <9> (10-1).
  \item The array access node's bound field is set to <11> ((40-10)/3 + 1).
  \item The array access node's stride field is set to <3>.
  \item Next, the section (3:9:2) is processed.
  \item The array access node's index field is set to <15> (9 + (3-1)*3).
  \item The array access node's bound field is set to <4> ((9-3)/2 + 1).
  \item The array access node's stride field is set to <6> (3 * 2).
  \end{enumerate}
  So, by adjusting the fields, one can more easily create the desired index
  set; in this case, {15, 21, 27, 33}.\\ \\



